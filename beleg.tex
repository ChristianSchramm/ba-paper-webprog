% **************************************************
% Clean Thesis
% -- A LaTeX Style for Thesis Documents --
%
% Copyright (C) 2011-2013 Ricardo Langner
% **************************************************
%
% Readme:
% ----------------------------------------
% *** Clean, Simple, Elegant ***
% "Clean Thesis" is a LaTeX style for thesis documents, developed
% for my diplom thesis (Diplomarbeit). The style can be understood
% as my personal compromise - a typical clean looking scientific
% document combined and polished with minor beautifications.
%
% The design of this "Clean Thesis" style is inspired
% by user guide documents from Apple Inc.
%
% Note: If you are looking for an exact and correct style regarding
% typographic rules, please have a look at the "Classic Thesis Style"
% (see http://www.miede.de/index.php?page=classicthesis).
%
% *** Donation = Postcard ***
% Based on the idea of Andr\'e Miede: If you like the "Clean Thesis"
% style I would be very pleased about a donation in the form of a
% POSTCARD. You can find my address in the file Clean-Thesis.pdf.
% I am going to collect all postcards and exhibit them at the website
% I mentioned.
%
% *** Idea and Inspiration ***
% The idea of providing my customized style for thesis documents
% passed through my mind while writing my own thesis. Motivated and
% inspired by the superb "Classic Thesis Style"
% (see http://www.miede.de/index.php?page=classicthesis) by Andr\'e Miede
% (thanks to Andr\'e for doing a great job) I decided to collect all
% design and style related functionality in a separate LaTeX style and
% provide this style to other thesis writers.
%
%
% License Information:
% ----------------------------------------
% "Clean Thesis" is free software: you can redistribute it and/or modify
% it under the terms of the GNU General Public License as published by
% the Free Software Foundation, either version 3 of the License, or
% (at your option) any later version.
%
% "Clean Thesis" is distributed in the hope that it will be useful,
% but WITHOUT ANY WARRANTY; without even the implied warranty of
% MERCHANTABILITY or FITNESS FOR A PARTICULAR PURPOSE.  See the
% GNU General Public License for more details.
%
% You should have received a copy of the GNU General Public License
% along with this program.  If not, see <http://www.gnu.org/licenses/>.
% **************************************************


% **************************************************
% Document Class Definition
% **************************************************
\documentclass[%
	paper=A4,					% paper size --> A4 is default in Germany
	twoside=false,				% onesite or twoside printing
	openright,					% doublepage cleaning ends up right side
	parskip=full,				% spacing value / method for paragraphs
	chapterprefix=true,			% prefix for chapter marks
	11pt,						% font size
	headings=normal,			% size of headings
	bibliography=totoc,			% include bib in toc
	listof=totoc,				% include listof entries in toc
	titlepage=on,				% own page for each title page
	captions=tableabove,		% display table captions above the float env
	draft=false,				% value for draft version
]{scrreprt}%

% **************************************************
% Debug LaTeX Information
% **************************************************
%\listfiles

% **************************************************
% Information and Commands for Reuse
% **************************************************
\newcommand{\thesisTitle}{Barrierefreiheitsdatenbank}
\newcommand{\thesisSubtitle}{Untersuchung ob es eine Möglichkeit gibt Daten und Bilder auf mobilen Endgeräten offline zu Erfassen und bei bestehender Internetverbindung mit dem Server abzugleichen}
\newcommand{\thesisName}{Christian Schramm}
\newcommand{\thesisMatrikel}{s3001102}
\newcommand{\thesisSubject}{Belegarbeit Webprogrammierung}
\newcommand{\thesisDate}{April 8, 2014}
\newcommand{\thesisVersion}{0.0.1}

\newcommand{\thesisFirstReviewer}{Maksim Gudow}
% \newcommand{\thesisFirstReviewerUniversity}{\protect{webit! Gesellschaft für neue Medien mbH}}
\newcommand{\thesisFirstReviewerDepartment}{BERUFSAKADEMIE SACHSEN Staatliche Studienakademie Dresden}

% \newcommand{\thesisSecondReviewer}{Prof. Eberhardt Engelhardt}
% \newcommand{\thesisSecondReviewerUniversity}{\protect{BERUFSAKADEMIE SACHSEN
% Staatliche Studienakademie Dresden}}
% \newcommand{\thesisSecondReviewerDepartment}{Department of Clean Thesis Style}

% \newcommand{\thesisFirstSupervisor}{Jane Doe}
% \newcommand{\thesisSecondSupervisor}{John Smith}

\newcommand{\thesisUniversity}{\protect{BERUFSAKADEMIE SACHSEN Staatliche Studienakademie Dresden}}
% \newcommand{\thesisUniversityDepartment}{}
% \newcommand{\thesisUniversityInstitute}{Institut for Clean Thesis Dev}
% \newcommand{\thesisUniversityGroup}{Clean Thesis Group (CTG)}
\newcommand{\thesisUniversityCity}{Dresden}
\newcommand{\thesisUniversityStreetAddress}{Hans-Grundig-Straße 25}
\newcommand{\thesisUniversityPostalCode}{01307}

% **************************************************
% Load and Configure Packages
% **************************************************
\usepackage[utf8]{inputenc}		% defines file's character encoding
\usepackage[ngerman]{babel} % babel system, adjust the language of the content
\usepackage[printonlyused,footnote,nohyperlinks]{acronym}

\usepackage{scrhack}

\usepackage[					% clean thesis style
	figuresep=colon,%
	sansserif=false,%
	hangfigurecaption=false,%
	hangsection=true,%
	hangsubsection=true,%
	colorize=full,%
	colortheme=bluemagenta,%
]{cleanthesis}

\hypersetup{					% setup the hyperref-package options
	pdftitle={\thesisTitle},	% 	- title (PDF meta)
	pdfsubject={\thesisSubject},% 	- subject (PDF meta)
	pdfauthor={\thesisName},	% 	- author (PDF meta)
	plainpages=false,			% 	-
	colorlinks=false,			% 	- colorize links?
	pdfborder={0 0 0},			% 	-
	breaklinks=true,			% 	- allow line break inside links
	bookmarksnumbered=true,		%
	bookmarksopen=true			%
}

% **************************************************
% Document CONTENT
% **************************************************
\begin{document}

% --------------------------
% rename document parts
% --------------------------
\renewcaptionname{ngerman}{\figurename}{Abb.}
\renewcaptionname{ngerman}{\tablename}{Tab.}
% \renewcaptionname{english}{\figurename}{Fig.}
% \renewcaptionname{english}{\tablename}{Tab.}

% --------------------------
% Front matter
% --------------------------
\pagenumbering{roman}			% roman page numbing (invisible for empty page style)
\pagestyle{empty}				% no header or footers
\input{Inhalt/Titelseite}		% INCLUDE: all titlepages
\cleardoublepage

% \pagestyle{plain}				% display just page numbers
% \input{Inhalt/abstract}		% INCLUDE: the abstracts (english and german)
% \cleardoublepage
%
% \input{Inhalt/acknowledgement} % INCLUDE: acknowledgement
% \cleardoublepage
%
\setcounter{tocdepth}{2}		% define depth of toc
\tableofcontents				% display table of contents
\cleardoublepage

% --------------------------
% Body matter
% --------------------------
\pagenumbering{arabic}			% arabic page numbering
\setcounter{page}{1}			% set page counter
\pagestyle{maincontentstyle} 	% fancy header and footer

% !TEX root = ../thesis-example.tex
%
\chapter{Einleitung}
\label{sec:Einleitung}

% \cleanchapterquote{The man who stops advertising to save money is like the man who stops the clock to save time.}{Thomas Jefferson}{(3ter Präsident der Vereinigten Staaten von Amerika)(1801-1809))} \\[20mm]

% Einleitungstext

% Die vorranschreitende Entwicklung des Internets der letzten Jahrzehnte und der flächendeckende Ausbau der Internetanbindung, hat nicht nur die Möglichkeiten der Informationsbeschaffung weitreichend verändert, auch die Möglichkeit der Warenbeschaffung unterliegt einem grundlegendem Umdenken. In immer mehr Bereichen verlagert sich der Einkauf von Waren zunehmend vom lokalen Einzelhandel auf \cite[22]{Jurgens:1995} überregionale Internetwarenhäuser, wie z. B. Amazon.


Das Thema Barrierefreiheit betrifft einen jährlich zunehmenden größer werdenden Teil der Bevölkerung. Im Jahr 2011 gab es mit knapp 7,3 Millionen schwerbehinderten Menschen rund 2,6\% mehr als noch im Jahr 2009.\cite[]{WEB:DESTATIS:2014} Dabei leiden zweidrittel der schwerbehinderten Menschen unter körperlichen Behinderungen, welche je nach Grad und Art der Behinderung zu sehr starken Einschränkungen im Alltag führen können.

Im Bereich des öffentlichen Lebens wird sehr viel dafür getan, um z.B. die Zugänge zu öffentlichen Einrichtungen wie Museen, Ämtern oder öffentlichen Verkehrsmitteln zu erleichtern. In der Verordnung DIN 18040 Barrierefreies Bauen sind grundlegende Anforderungen an öffentlich zugängliche Gebäude beschrieben. Die Einführung dieser Verordnung bzw. der einzelnen Punkte in die Technischen Baubestimmungen obliegt jedoch den einzelnen Bundesländern.\cite[]{WEB:DIN18040:2010}

Um erfassen zu können welche Einrichtungen die behindertengerechten Anforderungen erfüllen, wurde von der Thüringer Tourismus GmbH in Zusammenarbeit mit dem Dresdner Unternehmen webit! Gesellschaft für neue Medien mbH das Konzept der Barrierefreiheitsdatenbank entwickelt. Sie ermöglicht es alle Daten über eine Einrichtung zu erfassen. Dazu zählen z.B. Zugänge, Treppen, Liftanlagen und vieles mehr. Alle Daten die von Mitarbeitern des Unternehmens erfasst werden, kommen in einer Datenbank zusammen und sollen dabei helfen behinderten Menschen bei der Planung ihres Urlaubs oder von Ausflügen zu unterstützen.

\section{Motivation}
\label{sec:motivation:mot}

Die Überlegung zu dieser Arbeit kam durch die Erfassung der ersten öffentlich begehbaren Einrichtungen. Um zum derzeitigen Zeitpunkt Daten erfassen zu können, wird eine Internetverbindung benötigt. Diese kann aber bei vielen Einrichtungen, wie z.B. Kellergewölben, Museen oder Schlössern nicht gewährleistet werden. Dadurch entstand der Wunsch die Daten unabhängig von einer Internetverbindung einpflegen zu können. Die aufgenommenen Bilder und Informationen zu den baulichen Gegebenheiten sollen dann mit den vorhandenen Daten auf dem Server abgeglichen werden und gegebenenfalls auf dem Client, als auch auf dem Server ergänzt werden.

\section{Zielstellung und Abgrenzung der Arbeit}
\label{sec:goal:goal}

Smartphones und Tablets gehören mittlerweile zum Standard der mobilen Kommunikation. Mit diesen kleinen \"Alleskönnern\" kann man Fotos aufnehmen, im Internet surfen, Dokumente erfassen und vieles mehr, was noch vor ein paar Jahren nur von Desktoprechnern denkbar gewesen wäre. Das macht die heutige Datenerfassung sehr viel flexibler.

\section{Untersuchungsgegenstand}
\label{sec:investigation:inv}
 % INCLUDE: introduction
% !TEX root = ../thesis-example.tex
%
\chapter{Theorie/Begriffsbestimmung}
\label{sec:theorie:Theorie}

\cleanchapterquote{Most good programmers do programming not because they expect to get paid or get adulation by the public, but because it is fun to program.}{Linus Torvalds}{(Finnish American, software engineer and hacker)}

Zur Gewährleistung des allgemeinen Verständnisses, werden einige grundlegende Begriffe definiert, welche die Grundlage für die folgende Untersuchung bilden.

Die Entscheidung, welche mobile Variante für die Umsetzung verwendet werden soll, ist abhängig von den jeweiligen Eigenschaften der Anwendungen:

\begin{itemize}
	\item Wieviel Speicher steht der App zur Verfügung?
	\item Welche Handyfunktionen werden benötigt?
	\item Lässt sich feststellen wann eine Internetverbindung besteht?
	\item Lassen sich die Daten, abhängig von einer bestehenden Internetverbindung, mit dem Server abgleichen.
\end{itemize}

Die nachfolgenden Vor- und Nachteile von Nativen Apps, Web Apps und Hybrid Apps sind auf der Website www.app-entwickler-verzeichnis.de beschrieben.\cite[]{WEB:APPEV:2014}

\section{Native Apps}
\label{sec:intro:native}

Die Nativen Apps werden speziell für das jeweilige Betriebssystem entwickelt z.B. iOS oder Android. Diese laufen dann auch ausschließlich auf iOS Geräten wie dem iPhone und dem iPad, oder Android Geräten wie dem Samsung Galaxy S4.

Dadurch wird eine optimale Nutzung der Ressourcen und einheitlich funktionierende Hardwareschnittstellen sichergestellt.

\subsection{Vorteile}
\label{sec:native:pros}

\begin{itemize}

	\item Native Apps nutzen die Leistung des Betriebssystems und des verwendeten Gerätes voll aus, da sie speziell für das Betriebssystem angepasst sind. Dadurch lassen sich sehr gut komplexere und rechenintensivere Apps umsetzen.

	\item Durch die Installation der Apps auf dem Endgerät können Hardwarefunktionen wie Kamera, Beschleunigungssensor oder \ac{GPS} benutzt werden. Das ist in der Regel nur nativen Apps vorbehalten.

	\item Daten können auf dem Endgerät in beliebiger Menge gespeichert werden.

	\item Da Native Apps über einen Appstore vertrieben werden, wirken sich positive Bewertungen stark auf den Verkauf aus.

	\item Die App lässt sich sehr einfach über den Appstore installieren. Nach der Installation wird automatisch ein Icon zum Starten auf dem Homescreen angelegt.

	\item Der Vertriebsaufwand ist sehr gering, da die Appstores verbreitete Bezugsquellen für Native Apps sind. Ist die App erfolgreich, findet sich diese in den Top-Listen der App Stores wieder und erreicht dadurch sehr hohe Downloadzahlen.

\end{itemize}

\subsection{Nachteile}
\label{sec:native:cons}

\begin{itemize}

	\item Ein großer Nachteil ist der erforderliche Entwicklungsaufwand, wenn die App in allen Appstores angeboten werden soll, da die App an die jeweiligen Gegebenheiten des Betriebssystems optimiert werden muss.

	\item Es entstehen zusätzliche Kosten, um die App für den entsprechenden Appstore entwicklen und anbieten zu können.

\end{itemize}

\section{Web Apps}
\label{sec:intro:webapp}

Web Apps sind im eigentlichen Sinne speziell programmierte \ac{HTML5} Websites, die automatisch erkennen auf welchem Endgerät sie aufgerufen werden und optimieren den Inhalt entsprechend automatisch. Somit kann jedes mobile Endgerät, welches über einen Webbrowser verfügt, die App nutzen.

\subsection{Vorteile}
\label{sec:webapp:pros}

\begin{itemize}

	\item Web Apps sind quasi unabhängig vom Betriebssystem und funktionieren auf allen Smartphones. Dadurch werden mehr potentielle Nutzer bei gleichzeitig geringeren Kosten erreicht.

	\item In der Regel ist die Entwicklung einer Web App günstiger, als die Entwicklung einer nativen App für nur ein Betriebssystem.

	\item Durch die Verwendung von HTML5 wird auch die Offline-Speicherung von Daten ermöglicht. Somit kann nach erstmaligem laden die App auch ohne permanente Internetverbindung genutzt werden.

	\item Über Onlinesuchmaschinen wie beispielsweise Google können Web Apps ohne großen Aufwand gefunden und ohne Installation direkt genutzt werden. Werden diese als Lesezeichen gespeichert, lässt die Web App sich genau wie eine Native App vom Startbildschirm aus starten.

	\item Die Veröffentlichung und Aktualisierung einer Web App erfolgt in Sekundenschnelle, da sie im Gegensatz zu Nativen Apps keinen Zulassungsprozess durchlaufen muss.

	\item Vertreibt man die App selbst, entfällt die Provision von überlicherweise 30 Prozent an den Betreiber des App Stores.

\end{itemize}

\subsection{Nachteile}
\label{sec:webapp:cons}

\begin{itemize}

	\item Viele Hardwarefunktionen wie beispielsweise die Kamera oder das \ac{GPS} der mobilen Geräte lassen sich garnicht oder nur mit spezieller Zustimmung des Nutzers verwenden.

	\item Komplexe Berechnungen wie beispielsweise 3D Darstellungen, Verschlüsselungen oder Bildbearbeitungen sind mit einer Web App nicht möglich.

	\item Benötigt die App mehr als 10MB an Datenmaterial auf dem Endgerät, ist von einer Entwicklung als reine Web App abzusehen.

	\item Geschäftsmodelle, die auf In-App-Käufe oder einem App Store aufbauen, funktionieren zusammen mit der Web App nicht.

\end{itemize}

\section{Hybridapps}
\label{sec:intro:hybrid}

Hybridapps haben das Ziel, die Vorteile der Web App Entwicklung und der Entwicklung nativer Apps in sich zu vereinen. Dabei setzen die Entwickler auf eine große Anzahl von Frameworks. PhoneGap, Corona oder Appelerator Titanium sind Beispiele für Frameworks, mit deren Hilfe Web Apps in eine native App umgewandelt werden kann.

Hybrid Apps besitzen jedoch auch Vor- und Nachteile.

\subsection{Vorteile}
\label{sec:hybrid:pros}

\begin{itemize}

	\item Durch die Verwendung einer Hybrid App lässt sich eine Cross Browser Web App erstellen, die in allen modernen Browsern läuft.

	\item Da eine Web App mittels Frameworks für verschiedene Betriebssysteme umgewandelt werden kann, bleibt die eigenständige Entwicklung für jedes einzelne Betriebssystem erspart. Es bleiben im schlimmsten Fall nur betriebssystemspezifische Feinheiten, die noch angepasst werden müssen.

	\item Mit Javascript lassen sich viele Hardwarefunktionen der mobilen Endgeräte nutzen, auf die man bei einer Web App nicht zugreifen konnte.

	\item Der Verkauf einer Hybrid App erfolgt über den jeweiligen App Store.

\end{itemize}

\subsection{Nachteile}
\label{sec:hybrid:cons}

\begin{itemize}

	\item Ein großer Nachteil einer Hybrid App kann entsteht, wenn sehr rechenintensive Anwendungen verwendet werden. Dadurch erreichen Hybrid Apps sehr schnell ihr Leistungsmaximum reagieren träge. Dies ist sehr stark vom verwendeten Framework abhängig. Dieser Nachteil kann in Zukunft durch merklich effizienter werdende Frameworks behoben werden.

	\item Die Progammierung einer Hybrid App kann mit zunehmendem Komplexitätsgrad sehr aufwendig werden und dadurch eine Umsetzung mittels nativer App empfehlenswerter machen.

\end{itemize}

\section{Wahl der Appvariante}
\label{sec:intro:Appvariante}

Die Wahl der Appvariante ist nicht allein abhängig von den Vor- und Nachteilen der einzelnen Varianten. Hierfür muss auch die bestehende Website ``http://www.thueringen-tourismus.de/barrierefrei'' betrachtet und die Umsetzungsaufwände gegeneinander abgewägt werden.

Ein nicht zu unterschätzender Vorteil der Website ist die bereits responsive Umsetzung. (Abb.\ref{Barrierefreidatenbank}-\ref{Websitemobil}) Mit dieser passt sich der dargestellte Inhalt an die Auflösung des Endgeräts an, wodurch die optimale Darstellung auch auf mobilen Geräten gewährleistet wird. Es können somit bereits neue Daten für die Barrierefreiheitsdatenbank mobil erfasst werden.

\begin{figure}[htb]
	\includegraphics[width=\textwidth]{Bilder/Barrierefreidatenbank}
	\caption{Die Abbildung zeigt den Aufbau der Website http://www.thueringen-tourismus.de/barrierefrei auf der Daten für die Barrierefreiheitsdatenbanke eingepflegt und abgerufen werden können.}
	\label{Barrierefreidatenbank}
\end{figure}

\begin{figure}[htb]
	\includegraphics[width=\textwidth]{Bilder/Datenerfassung}
	\caption{In der Abbildung ist ein Standardformular zu sehen, welches grundlegende Eigenschaften für die Datenerfassung einer Einrichtung erhält. Wie beispielsweise die Art der Einrichtung.}
	\label{Datenerfassung}
\end{figure}

\begin{figure}[htb]
	\begin{tabular}{l r}
		\includegraphics[width=0.49\textwidth]{Bilder/Barrierefreidatenbank-mobil}
		&
		\includegraphics[width=0.49\textwidth]{Bilder/Datenerfassung-mobil}
	\end{tabular}
	\caption{Auf den beiden oberen Bilder ist sowohl die Ausgabe der erfassten Daten, als auch die Seite zur Datenerfassung auf einem mobilen Endgerät zu sehen. Navigation und Inhaltselemente passen sich an die Auflösung des Endgeräts an.}
	\label{Websitemobil}
\end{figure}

\cleardoublepage

Nach Betrachtung der Vor- und Nachteile der drei App Varianten und der bestehenden Website, bietet sich die Umsetzung einer reinen nativen App für die Datenerfassung der Barrierefreiheitsdatenbank nicht an. Der Umsetzungsaufwand und die entstehenden Kosten wären unangemessen hoch im Vergleich zu den daraus entstehenden Vorteilen.

Die responsive Variante der Website bietet sich auch nicht an, da sie eine permanente Internetverbindung voraussetzt.

Da eine Web App auf \ac{HTML5} aufbaut und die Website sich bereits teilweise an das Endgerät anpasst, baut die Untersuchung im zweiten Teil der Arbeit auf eine Umsetzung als Web App auf. Daher wird bei der Synchronisation der Daten zur Datenbank auf \ac{HTML5} und Javascript gesetzt.

Eine Hybrid App baut auf einer Web App auf und bietet sich daher für eine Weiterentwicklung der Lösung an. Dies kann aber zu einem späteren Zeitpunkt erfolgen und wird daher vorerst nicht betrachtet.
 % INCLUDE: Grundlagen
% !TEX root = ../thesis-example.tex
%
\chapter{Synchronisation}
\label{sec:Synchronisation}

\cleanchapterquote{The most important property of a program is whether it accomplishes the intention of its user.}{C.A.R. Hoare}{(British computer scientist,\\ winner of the 1980 Turing Award)}

Synchronisation beschreibt den Datenaustausch zwischen einem Sender und einem Empfänger. Dabei werden die Daten in Blöcke aufgeteilt und in einen Übertragungsrahmen eingepasst. Um die Daten aneinander anzugleichen muss festgestellt werden, welches Endgerät welche Daten besitzt und muss kontrolliert werden ob das andere Gerät eine Anforderung für diese Daten besitzt.

Besitzen beide Endgeräte dieselben Daten z.B. in unterschiedlichen Versionen, muss definiert werden, wie mit den Änderungen umgegangen werden soll.\cite[]{WEB:SYNCML:2014}

Aufgrund der Begrenzung des lokalen Gerätespeichers empfiehlt sich die Synchronisation auf zwei seperaten Wegen durchzuführen. Als Erstes wird die Datenbank synchronisiert und als Zweites alle Anhänge, wie z.B. Bilder oder Dokumente. So kann das Problem der Speicherbegrenzung bei Web Apps umgangen werden.

\section{Datenbanksynchronisation am Beispiel ``WebSqlSync''}
\label{sec:dbsync}

WebSqlSync\footnote{http://www.verious.com/code/orbitaloop/WebSqlSync/} ist eine Javascript Bibliothek zur automatischen Synchronisation einer lokalen WebSql Datenbank mit dem Server. Die Synchronisation erfolgt dabei in beide Richtungen und arbeitet auf dem Prinzip der inkrementellen Synchronisation, was bedeutet, dass nur erforderliche Daten übertragen werden.

WebSqlSync funktioniert auch ohne Internetverbindung. Alle Änderungen der Daten werden dabei verfolgt und mit dem Server abgeglichen, sobald wieder eine Internetverbindung besteht. Es wird auch die Änderung der Daten auf mehreren Geräten unterstützt.

Die Unterstützung von Webapp und der Phonegap App\footnote{http://phonegap.com/} für mobile Betriebssysteme wie z.B. iOS und Android ermöglicht eine einfache Integration ohne den Programmcode anpassen zu müssen.\cite[]{WEB:WEBSQLSYNC:2014}

\textbf{Installation und Initialisierung}

Um WebSqlSync nutzen zu können, muss nur die Datei webSqlSync.js im Head-Bereich im \ac{HTML} des Projekts hinzugefügt werden.

\lstset{language=html}
\lstinline$<script src="lib/webSqlSync.js" type="application/x-javascript" charset="utf-8"></script>$

Bei Aufruf der Bibliothek, werden automatisch zwei Datenbanktabellen erstellt, falls diese nicht bereits durch einen vorherigen Aufruf existieren. Die erste Tabelle \textbf{\lstinline$new_elem$} speichert alle neuen bzw. geänderten Elemente, die zweite Tabelle \textbf{\lstinline$sync_info$} das Datum der letzten Synchronisation.

Zusätzlich werden sogenannte SQLite Auslöser erstellt, die überwachen, ob Änderungen per \textbf{\lstinline$INSERT$} oder \textbf{\lstinline$UPDATE$} an den Tabellen vorgenommen wurden. SQLite ist eine einfache Datenbankbibliothek, die Befehle der Sprache \ac{SQL} verwendet.

Geänderte Elemente werden somit automatisch in der Tabelle \textbf{\lstinline$new_elem$} eingefügt.(Abb.\ref{code:initsync})

\begin{figure}[htb]
	\lstset{language=html}
	\begin{lstlisting}
	DBSYNC.initSync(
		TABLES_TO_SYNC, webSqlDb, sync_info,
		'http://www.myserver.com', callBackEndInit
	);\end{lstlisting}
	\caption{Codebeispiel für den Aufruf zur Datenbanksynchronisation}
	\label{code:initsync}
\end{figure}

\hspace{1 cm}

Die Tabellen die, mit dem Server synchronisiert werden sollen, werden in der Funktion \textbf{\lstinline$TABLES_TO_SYNC$} angegeben. (Abb.\ref{code:tabletosync})

\begin{figure}[htb]
	\lstset{language=html}
	\begin{lstlisting}
	TABLES_TO_SYNC = [
	  {tableName : 'table1', idName : 'the_id'},
	  {tableName : 'table2'}
	  //if idName not specified, it will assume that it's "id"
	];
	\end{lstlisting}
	\caption{Codebeispiel für die Datenbanktabellen die synchronisiert werden sollen}
	\label{code:tabletosync}
\end{figure}

\hspace{1 cm}

In der Tabelle \textbf{\lstinline$sync_info$} können alle Informationen gespeichert werden, die der Entwickler als nützlich empfindet, beispielsweise die Identifikation des Clients, da Sie mit an den Server gesendet wird. Dafür kann jegliche Information genutzt werden, wie z.B. die Emailadresse, ein Login oder auch eine entsprechende \ac{ID} des genutzten mobilen Endgeräts.

\textbf{Aufruf}

Um die Synchronisation zu starten wird die Funktion \textbf{\lstinline$syncNow$} aufgerufen. Die Synchronisation erfolgt dabei nach einer freiwählbaren Zeitspanne oder aber nach einer festgelegten Anzahl von Datenänderungen. (Abb.\ref{code:syncnow})

\begin{figure}[htb]
	\lstset{language=html}
	\begin{lstlisting}
	DBSYNC.syncNow(callBackSyncProgress, function(result) {
	  if (result.syncOK === true) {
	    //Synchronized successfully
	  }
	});
	\end{lstlisting}
	\caption{Codebeispiel für den Aufruf zur Synchronisation}
	\label{code:syncnow}
\end{figure}

\hspace{1 cm}

Bei größeren Datenmengen ist es für den Nutzer hilfreich, wenn dieser eine Fortschrittsanzeige erhält. Während der Synchronisation wird dafür bei jedem Einzelschritt, beispielsweise einzelne Datenpakete, die Funktion \textbf{\lstinline$callBackSyncProgress$} aufgerufen. (Abb.\ref{code:syncprocess})

\begin{figure}[htb]
	\lstset{language=html}
	\begin{lstlisting}
	callBackSyncProgress: function(message, percent, msgKey) {
	  $('#uiProgress').html(message+' ('+percent+'%)');
	},
	\end{lstlisting}
	\caption{Codebeispiel für den Callback zur Erstellung einer Fortschrittsanzeige}
	\label{code:syncprocess}
\end{figure}

\hspace{1 cm}

\textbf{Einschränkungen}

Die Bibliothek WebSqlSync besitzt auch ein paar wenige Einschränkungen. Zum Beispiel wird der \ac{SQL}-Befehl \textbf{\lstinline$DELETE$} nicht unterstützt. Stattdessen sollte dies mit einem update an der entsprechenden Stelle umgangen werden.

\section{Dateisynchronisation am Beispiel ``ownCloud''}
\label{sec:datasync}

ownCloud\footnote{https://owncloud.com/} ist eine Open Source Software zur Einrichtung einer unabhängigen serverseitigen Datenspeicherlösung. Sogenannte Cloudspeicher gibt es mittlerweile sehr viele. Die wohl bekanntesten sind Dropbox, Google Drive und OneDrive. Sie ermöglichen einen einfachen Datenzugriff von überall auf der Welt. Dafür lädt man seine Dateien über eine Internetverbindung auf einen speziellen Datenserver. Leider besitzt dies den großen Nachteil, dass mkeine 100 prozentige Sicherheit besteht, was mit den eigenen Daten passiert.

Um dies sicherzustellen bietet ownCloud die Möglichkeit einen eigenen Cloudserver\ref{Architektur} zu erstellen, auf dem der Nutzer alles selbst konfigurieren kann. Per Nutzer- und Rechteverwaltung lässt sich wie auf einem normalen Server festlegen, welcher Nutzer Dateien verändern kann.

Für die Daten werden Backup Lösungen angeboten, so dass jederzeit Sicherungen der Dateien angelegt werden können. Der wohl wichtigste Punkt ist jedoch die Verschlüsselung der gespeicherten Daten mittels \ac{SSL}. Dadurch können auch unternehmensrelevante Daten sicher gespeichert werden.

Durch die Open Source Lösung ownCloud lässt sich zu den bereits vorhandenen Plugins neue Software entwickeln und den eigenen Bedürfnissen anpassen. Da bereits Apps für die mobilen Betriebssystem (Abb.\ref{ownCloudApp}) zu Verfügung stehen, lassen sich relativ einfach Daten zwischen dem Server und dem Client synchronisieren.

Neben der frei verfügbaren ``Community Edition'' stehen für die volle Unterstützung der ownCloud Entwickler, welche bei Fragen zur Software oder den Apps zur Verfügung stehen, auch eine ``Business'' und eine ``Enterprise Version'' zur Verfügung, welche gegen eine jährliche Gebühr angeboten wird.\cite[]{WEB:OWNCLOUD:2014}

\hspace{2 cm}

\begin{figure}[htb]
	\includegraphics[width=\textwidth]{Bilder/architecture}
	\caption{Architektur ownCloud}
	\label{Architektur}
\end{figure}

\hspace{2 cm}

\textbf{Vorteile der Android/iOS App von ownCloud}

Die nachfolgenden Vor- und Nachteile der ownCloud App sind auf der Website https://owncloud.com/ beschrieben.\cite[]{WEB:OWNCLOUD:2014}

\begin{itemize}

	\item SSL- und HTTP-Verbindungen werden automatisch erkannt, so dass eine einfache und gesicherte Verbindung zu dem Server möglich wird.

	\item Dateien und Ordner auf dem Server können neu angelegt, durchsucht, umbenannt und gelöscht werden, je nachdem wie die Rechte an den Nutzer vergeben sind.

	\item Durch das Anlegen von Favoriten lassen sich Daten wie Dokumente, Bilder und Videos automatisch mit dem Server synchronisieren.

	\item Dateien können auf das mobile Endgerät heruntergeladen werden, um diese Offline nutzen zu können.

	\item Es werden durch die App mehrere ownCloud Accounts mit einem Device und zusätzlich die Anbindung an mehrere ownCloud Server unterstützt.

\end{itemize}

\hspace{2 cm}

\begin{figure}[htb]
	\begin{tabular}{l r}
		\includegraphics[width=0.49\textwidth]{Bilder/ownCloud-mobile1}
		&
		\includegraphics[width=0.49\textwidth]{Bilder/ownCloud-mobile2}
	\end{tabular}
	\caption{ownCloud App}
	\label{ownCloudApp}
\end{figure}

\hspace{2 cm}

Unabhängig von der Anzahl der Nutzer lassen sich mit dieser Cloudlösung auch die für die Datenbank aufgenommenen Bilder unkompliziert speichern. Egal ob ein Internetzugang zur Verfügung steht oder man Fotos offline aufgenommen und in einem Ordner von ownCloud ablegt werden, synchronisiert ownCloud diese automatisch sobald eine Internetverbindung zur Verfügung steht.

Einziger Nachteil ist dabei der hohe Kostenfaktor für die ``Enterprise'' Lösung, den man mit dem Kunden abstimmen muss. In Ausblick auf die Erweiterbarkeit durch die Open Source \ac{API} lässt sich die Software beliebig erweitern und kann somit den Nutzen-Kosten-Faktor verringern.
 % INCLUDE: Synchronisationssoftware

% \input{Inhalt/chapter-related-work} % INCLUDE: related work
% \input{Inhalt/chapter-system}	% INCLUDE: system
% \input{Inhalt/chapter-concepts} % INCLUDE: concepts
% !TEX root = ../thesis-example.tex
%
\chapter{Fazit}
\label{sec:Fazit}

Zu Beginn der Arbeit werden die zur Verfügung stehenden Varianten der Umsetzung der Barrierefreiheitsdatenbank auf mobilen Endgeräten betrachtet, da sich diese stark auf die Möglichkeiten der Datenspeicherung und der Synchronisation auswirken. Auch die bereits bestehende Website wird für die Wahl der besten Variante herangezogen und beeinflusst maßgeblich die Wahl der Synchronisation.

Eine kurze Erläuterung der Funktionen der beiden Lösungsansätze soll vermitteln, wie diese für die Barrierefreiheitsdatenbank eingesetzt werden kann und welche Kombinationsmöglichkeiten es gibt.

Die Programmbibliothek WebSqlSync zur Datenbanksynchronisation und die Dateispeicherlösung ownCloud sind nur zwei Beispiele, wie die Synchronisation zwischen einem mobilen Endgerät und einem Server gewährleisten werden kann. Mit der Umsetzung einer Web App und der Möglichkeit diese mit Hilfe eines zusätzlichen Javascript-Frameworks in eine Hybrid App umwandeln zu können, lassen sich beide Lösungen mit etwas mehr Aufwand zu einer Anwendung zusammenführen.

\section{Auswertung}
\label{sec:Fazit:Auswertung}

Nach Betrachtung der beiden Beispiele lässt sich sagen, dass Hypothese eins[\ref{subsec:hypothesis:hypo}] und drei[\ref{subsec:hypothesis:hypo}] sich bewahrheitet haben. Durch die Trennung von Datenbanksynchronisation und Dateisynchronisation lassen sich Beschränkungen, die bei einer Web App an den lokalen Speicher der mobilen Endgeräte bestehen umgehen. Eine Web App bietet sich als Plattform für mobile Geräte an, da sie auf das bereits bestehende Grundgerüst der Barrierefreiheitsdatenbank aufbaut. Gleichzeitig erspart sich so viel Aufwand was zur Folge hat, dass sich anfallende Umsetzungskosten verringern, im Gegensatz zu der Entwicklung einer nativen App.

Beide Lösungen bieten die Möglichkeit auch ohne Internetzugang weiter Daten einpflegen zu können und erst bei bestehender Internetverbindung diese zu synchronisieren. Auch bei einem erneuten Verbindungsabbruch bleiben die Daten erhalten und werden später einfach erneut synchronisiert.

Hypothese zwei[\ref{subsec:hypothesis:hypo}] hat sich nach der Betrachtung als eine weniger optimale Lösung herausgestellt. Der Aufwand und die Kosten stehen in keinem Verhältnis zum Vorteil des erhöhten Speichergewinns und der Kombination aus Datei- und Datenbankspeichers.

\section{Ausblick}
\label{sec:Fazit:Ausblick}

Im Ausblick auf die Weiterentwicklung der Barrierefreiheitsdatenbank und einer größer werdenden Zielgruppe, bietet der Ansatz der getrennten Synchronisationswege einige Vorteile. Es lassen sich beliebig Nutzer ergänzen, die neue Daten einpflegen können. Der Speicherbedarf für die Datenbank ist begrenzt, da nur Text in der Datenbank gespeichert wird. Bilder werden in einem seperaten Ordner im Endgerät gespeichert, dessen Speicherplatz nur durch die Größe der Speicherkarte des Endgeräts begrenzt wird.

Eine Weiterentwicklung des Lösungsansatzes kann die Umwandlung der Web App in eine Hybrid App sein. Dadurch ergibt sich die Möglichkeit die Cloudfunktionen und die Datenbanksynchronisation in einer App zusammenzuführen und zusätzlich Hardwarefunktionen wie z.B. die Kamera zu integrieren. Der erweiterte Speicher ist für die damit in Verbindung stehende Projektkomplexität von Vorteil.

Der nächste Schritt ist die Erstellung einer Prototyp Anwendung, um die Funktionen am praktischen Beispiel zu testen.
 % INCLUDE: conclusion
\cleardoublepage

% --------------------------
% Back matter
% --------------------------
{%
\setstretch{1.1}
\renewcommand{\bibfont}{\normalfont\small}
\setlength{\biblabelsep}{0pt}
\setlength{\bibitemsep}{0.5\baselineskip plus 0.5\baselineskip}
\nocite{*}
% \printbibliography[nottype=online]
\printbibliography[heading=subbibliography,title={Webseiten},type=online,prefixnumbers={@}]
}
\cleardoublepage

\listoffigures
\cleardoublepage

% \listoftables
% \cleardoublepage

% % !TEX root = ../thesis-example.tex
%
\chapter{Abkürzungsverzeichnis}
\label{sec:acronym}

\begin{acronym}[laaaaaaaaaaaaaaaaang]
	\setlength{\itemsep}{-\parsep}

	\acro{GPS}{Global Positioning System}
	\acro{HTML}{Hypertext Markup Language}
	\acro{HTML5}{Hypertext Markup Language Version 5}
	\acro{SQL}{Structured Query Language}
	\acro{ID}{Ausweis, Kennnummer}
	\acro{jQuery}}{freie Javascript Bibliothek}
	% \acro{SEM}[Search Engine Marketing]

\end{acronym}
 % INCLUDE: Abkürzungsverzeichnis
% \cleardoublepage
% \input{Inhalt/colophon}

% !TEX root = ../thesis-example.tex
%
%************************************************
% Declaration
%************************************************
\pdfbookmark[0]{Selbstständigkeitserklärung}{Selbstständigkeitserklärung}
\chapter*{Selbstständigkeitserklärung}
\label{sec:Selbstständigkeitserklärung}
\thispagestyle{empty}

Ich, \thesisName, Matrikel-Nr.\ \thesisMatrikel, versichere hiermit, dass ich meinen Praxistransferbeleg mit dem Thema
\begin{quote}
\textit{\thesisTitle} - \textit{\thesisSubtitle}
\end{quote}
selbstst\"{a}ndig verfasst und keine anderen als die angegebenen Quellen und Hilfsmittel benutzt habe, wobei ich alle w\"{o}rtlichen und sinngemäßen Zitate als solche gekennzeichnet habe. Die Arbeit wurde bisher keiner anderen Pr\"{u}fungsbeh\"{o}rde vorgelegt und auch nicht ver\"{o}ffentlicht.

\bigskip

\noindent\textit{\thesisUniversityCity, \thesisDate}

\smallskip

\begin{flushright}
	\begin{minipage}{5cm}
		\rule{\textwidth}{1pt}
		\centering\thesisName
	\end{minipage}
\end{flushright}

%*****************************************
%*****************************************

\clearpage
\newpage
\mbox{}

% **************************************************
% End of Document CONTENT
% **************************************************
\end{document}
