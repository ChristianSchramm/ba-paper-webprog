% !TEX root = ../thesis-example.tex
%
\chapter{Einleitung}
\label{sec:Einleitung}

\cleanchapterquote{There are two ways of constructing a software design: One way is to make it so simple that there are obviously no deficiencies, and the other way is to make it so complicated that there are no obvious deficiencies. The first method is far more difficult.}{C.A.R. Hoare}{(British computer scientist,\\winner of the 1980 Turing Award)}

Das Thema Barrierefreiheit betrifft einen jährlich zunehmend größer werdenden Teil der Bevölkerung. Im Jahr 2011 gab es mit knapp 7,3 Millionen schwerbehinderten Menschen rund 2,6 Prozent mehr als noch im Jahr 2009 \cite[]{WEB:DESTATIS:2014}. Dabei leiden zweidrittel der schwerbehinderten Menschen unter körperlichen Behinderungen, die je nach Art und Grad der Behinderung zu sehr starken Einschränkungen im Alltag führen können.

Im Bereich des öffentlichen Lebens wird sehr viel für die Aufhebunng der Einschränkungen, um z.B. die Zugänge zu öffentlichen Einrichtungen wie Museen, Ämtern oder öffentlichen Verkehrsmitteln zu erleichtern. In der Verordnung DIN 18040 \cite[]{WEB:DIN18040:2010} Barrierefreies Bauen sind grundlegende Anforderungen an öffentlich zugängliche Gebäude beschrieben. Die Einführung dieser Verordnung bzw. der einzelnen Punkte in die Technischen Baubestimmungen obliegt jedoch den einzelnen Bundesländern.

Um zu erfassen, welche Einrichtungen die behindertengerechten Anforderungen erfüllen, wurde von der Thüringer Tourismus GmbH in Zusammenarbeit mit dem Dresdner Unternehmen webit! Gesellschaft für neue Medien mbH das Konzept der Barrierefreiheitsdatenbank entwickelt. Sie ermöglicht es alle Daten über eine Einrichtung zu erfassen. Dazu zählen z.B. Zugänge, Treppen, Liftanlagen und vieles mehr. Alle Daten, die von Mitarbeitern des Unternehmens erfasst werden, kommen in einer Datenbank zusammen und sollen behinderten Menschen bei der Planung ihres Urlaubs oder Ausflügen helfen.

\section{Motivation}
\label{sec:motivation:mot}

Die Idee dieser Arbeit entwickelte sich bei der Erfassung der ersten öffentlich begehbaren Einrichtungen. Um Daten erfassen zu können, wird eine Internetverbindung benötigt. Diese kann aber bei vielen Einrichtungen, wie z.B. Kellergewölben, Museen oder Schlössern nicht gewährleistet werden. Die baulichen Gegebenheiten verhindern häufig eine stabile Funkverbindung zwischen dem mobilen Endgerät und dem Server. Daher entstand der Wunsch die Daten unabhängig von einer Internetverbindung einpflegen zu können. Die aufgenommenen Bilder und Informationen zu den baulichen Gegebenheiten sollen dann mit den vorhandenen Daten auf dem Server abgeglichen und sowohl auf dem Client, als auch auf dem Server ergänzt werden.

\section{Zielstellung und Abgrenzung der Arbeit}
\label{sec:goal:goal}

Smartphones und Tablets gehören mittlerweile zum Standard der mobilen Kommunikation. Mit diesen kleinen ``Alleskönnern'' kann man Fotos aufnehmen, im Internet surfen, Dokumente erfassen und vieles mehr, was noch vor ein paar Jahren nur von Desktoprechnern aus möglich war. Das macht die heutige Datenerfassung sehr viel flexibler als noch vor ein paar Jahren.

Ziel der Arbeit ist es eine Möglichkeit zu finden, die mit mobilen Endgeräten erfassen Daten mit den Daten auf einem Server zu synchronisieren. Dabei soll sowohl die Datenbank, als auch Bilddateien betrachtet werden. Als Ausgangspunkt dienen die Vor- und Nachteile, die es bei nativen Apps, Web Apps und einer Hybrid App Variante gibt und wie sich daraus entstehenden Einschränkungen auf eine Synchronisation der Daten auswirken.

In dieser Arbeit liegt der Fokus nicht auf der Programmierung einer nativen App oder Web App. Stattdessen werden Überlegungen zur Umsetzung einer erfolgreichen Synchronisation der Daten angestellt, welche die Wahl einer mobilen Variante der Barrierefreiheitsdatenbank beeinflussen können.

\section{Untersuchungsgegenstand}
\label{sec:investigation:inv}

Im folgenden Abschnitt werden unterschiedliche Methoden der Datensynchronisation betrachtet und die Vor- und Nachteile miteinander verglichen, um für das Kundenprojekt einen Lösungsansatz zu finden. Die Ergebnisse sollen dabei helfen, die nachfolgende Forschungsfrage zu widerlegen oder zu bestätigen.

\subsection{Forschungsfrage}
\label{subsec:problem:prob}

Aus der Zielstellung ergibt sich eine Forschungsfrage, die mit Hilfe der zu erarbeitenden Ergebnisse am Ende dieser Arbeit beantwortet wird.

\begin{quote}
	\textbf {Gibt es eine Möglichkeit alle Daten, die mit mobilen Endgeräten offline erfasst werden, bei bestehender Internetverbindung mit dem Server abzugleichen?}
\end{quote}

\cleardoublepage

\subsection{Hypothesen}
\label{subsec:hypothesis:hypo}

Daraus ergeben sich folgende Hypothesen, die sich bei der Untersuchung der Methoden der Datenbank- und Dateisynchronisation von mobilen Endgeräten mit einem Server als richtig oder falsch herausstellen werden.

\textbf {1. Hypothese}

Es lassen sich alle mit einem mobilen Endgerät erfassten Daten uneingeschränkt per Internet mit dem Server synchronisieren.

\textbf {2. Hypothese}

Die Entwicklung einer nativen App ist die beste Lösung für die Nutzung der Hardwarefunktionen und des Speicherbedarfs der Kombination aus Datenbank und Bildern.

\textbf {3. Hypothese}

Die Umsetzung einer Web App bietet sich aufgrund der bestehenden Website an. Dadurch verringern sich Aufwand und Kosten, ohne Einschränkung der Funktionen.
