% !TEX root = ../thesis-example.tex
%
\chapter{Fazit}
\label{sec:Fazit}

Zu Beginn der Arbeit werden die zur Verfügung stehenden Varianten der Umsetzung der Barrierefreiheitsdatenbank auf mobilen Endgeräten betrachtet, da sich diese stark auf die Möglichkeiten der Datenspeicherung und der Synchronisation auswirken. Auch die bereits bestehende Website wird für die Wahl der besten Variante herangezogen und beeinflusst maßgeblich die Wahl der Synchronisation.

Eine kurze Erläuterung der Funktionen der beiden Lösungsansätze soll vermitteln, wie diese für die Barrierefreiheitsdatenbank eingesetzt werden kann und welche Kombinationsmöglichkeiten es gibt.

Die Programmbibliothek WebSqlSync zur Datenbanksynchronisation und die Dateispeicherlösung ownCloud sind nur zwei Beispiele, wie die Synchronisation zwischen einem mobilen Endgerät und einem Server gewährleisten werden kann. Mit der Umsetzung einer Web App und der Möglichkeit diese mit Hilfe eines zusätzlichen Javascript-Frameworks in eine Hybrid App umwandeln zu können, lassen sich beide Lösungen mit etwas mehr Aufwand zu einer Anwendung zusammenführen.

\section{Auswertung}
\label{sec:Fazit:Auswertung}

Nach Betrachtung der beiden Beispiele lässt sich sagen, dass Hypothese eins[\ref{subsec:hypothesis:hypo}] und drei[\ref{subsec:hypothesis:hypo}] sich bewahrheitet haben. Durch die Trennung von Datenbanksynchronisation und Dateisynchronisation lassen sich Beschränkungen, die bei einer Web App an den lokalen Speicher der mobilen Endgeräte bestehen umgehen. Eine Web App bietet sich als Plattform für mobile Geräte an, da sie auf das bereits bestehende Grundgerüst der Barrierefreiheitsdatenbank aufbaut. Gleichzeitig erspart sich so viel Aufwand was zur Folge hat, dass sich anfallende Umsetzungskosten verringern, im Gegensatz zu der Entwicklung einer nativen App.

Beide Lösungen bieten die Möglichkeit auch ohne Internetzugang weiter Daten einpflegen zu können und erst bei bestehender Internetverbindung diese zu synchronisieren. Auch bei einem erneuten Verbindungsabbruch bleiben die Daten erhalten und werden später einfach erneut synchronisiert.

Hypothese zwei[\ref{subsec:hypothesis:hypo}] hat sich nach der Betrachtung als eine weniger optimale Lösung herausgestellt. Der Aufwand und die Kosten stehen in keinem Verhältnis zum Vorteil des erhöhten Speichergewinns und der Kombination aus Datei- und Datenbankspeichers.

\section{Ausblick}
\label{sec:Fazit:Ausblick}

Im Ausblick auf die Weiterentwicklung der Barrierefreiheitsdatenbank und einer größer werdenden Zielgruppe, bietet der Ansatz der getrennten Synchronisationswege einige Vorteile. Es lassen sich beliebig Nutzer ergänzen, die neue Daten einpflegen können. Der Speicherbedarf für die Datenbank ist begrenzt, da nur Text in der Datenbank gespeichert wird. Bilder werden in einem seperaten Ordner im Endgerät gespeichert, dessen Speicherplatz nur durch die Größe der Speicherkarte des Endgeräts begrenzt wird.

Eine Weiterentwicklung des Lösungsansatzes kann die Umwandlung der Web App in eine Hybrid App sein. Dadurch ergibt sich die Möglichkeit die Cloudfunktionen und die Datenbanksynchronisation in einer App zusammenzuführen und zusätzlich Hardwarefunktionen wie z.B. die Kamera zu integrieren. Der erweiterte Speicher ist für die damit in Verbindung stehende Projektkomplexität von Vorteil.

Der nächste Schritt ist die Erstellung einer Prototyp Anwendung, um die Funktionen am praktischen Beispiel zu testen.
